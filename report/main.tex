\documentclass[a4paper, 14pt]{article}
\usepackage{pre}


\begin{document}
    % Оформление титульного листа
\begin{titlepage}
    \newpage
    
    \begin{center}
    МИНИСТЕРСТВО НАУКИ И ВЫСШЕГО ОБРАЗОВАНИЯ РОССИЙСКОЙ ФЕДЕРАЦИИ ФЕДЕРАЛЬНОЕ ГОСУДАРСТВЕННОЕ АВТОНОМНОЕ ОБРАЗОВАТЕЛЬНОЕ УЧРЕЖДЕНИЕ ВЫСШЕГО ОБРАЗОВАНИЯ НАЦИОНАЛЬНЫЙ ИССЛЕДОВАТЕЛЬСКИЙ ЯДЕРНЫЙ УНИВЕРСИТЕТ «МИФИ» \\
    \end{center}
    
    
    \vspace{2.5em} % Делаем вертикальный пробел в 3em
    
    \begin{center}
        \textsc{\textbf{Кафедра теоретической ядерной физики\\}}
    \end{center}
    \vspace{1em}
    \begin{center}
    \textsc{На правах рукописи\\} 
    \end{center}
    \begin{center}
        \textsc{\Large Широков Денис Дмитриевич\linebreak  \linebreak  \Large \textbf{ <<Численные методы решения задач математической физики с использованием локально-адаптивных сеток>>}}
    \end{center} 
    
    \vspace{1em}
    
    \begin{center}
    Выпускная квалификационная работа бакалавра \linebreak Направление подготовки 03.03.01 Прикладные математика и физика
    \end{center}
    
    \vspace{1em}
    
   
    \newbox{\lbox}
    \savebox{\lbox}{\hbox{Широков Денис Дмитриевич}}
    \newlength{\maxl}
    \setlength{\maxl}{\wd\lbox}
    \hfill\parbox{9cm}{
    \hspace*{1.5cm}\hspace*{-3cm}Выпускная квалификационная\\ \hspace*{1.5cm}\hspace*{-3cm}работа защищена\\ 
    
    \hspace*{1.5cm}\hspace*{-3cm}<<\rule{1.5em}{1pt}>>\hbox to\maxl{\rule{6em}{1pt}2021 г.\hfill}\\
    
    \hspace*{1.5cm}\hspace*{-3cm}Оценка \hbox to\maxl{\rule{7em}{1pt}\hfill}\\
    
    \hspace*{1.5cm}\hspace*{-3cm}Секретарь ГЭК 
    \hbox to\maxl{\rule{5em}{1pt} Фамилия И.О.\hfill}\hfill\\  
    \hspace*{7.0cm}\hspace*{-3cm}к.ф.-м.н., доцент\hfill\\
    
    }
    
    

    
    
    \vspace{\fill}
    
    \begin{center}
    \textbf{Москва}\\
    \today
    \end{center}

\end{titlepage}

    % Оформление титульного листа
\begin{titlepage}
    \newpage
    
    \vspace{3em} % Делаем вертикальный пробел в 3em
    
   
    \begin{center}
        \textsc{ \textbf{Пояснительная записка\linebreak к бакалаврской дипломной работе: \linebreak  \Large <<Численное решение уравнения теплопроводности с использованием локально-адаптивных сеток>>}}
    \end{center} 

    
    \vspace{5em}
    
   
    \newbox{\lbox}
    \savebox{\lbox}{\hbox{Широков Денис Дмитриевич}}
    
    \setlength{\maxl}{\wd\lbox}
    \hfill\parbox{12cm}{
    \hspace*{1cm}\hspace*{-1cm}Студент\hfill\hbox to\maxl{\rule{5em}{1pt} Широков Д.Д.\hfill}\\
    
    \hspace*{1cm}\hspace*{-1cm}Научный руководитель\hfill\\
    \hspace*{1cm}\hspace*{-1cm}к.ф.-м.н.\hfill\hbox to\maxl{\rule{5em}{1pt} Кучугов П.А.\hfill}\\ \\
    \hspace*{1cm}\hspace*{-1cm}Рецензент\hfill\\
    \hspace*{1cm}\hspace*{-1cm}к.ф.-м.н.\hfill\hbox to\maxl{\rule{5em}{1pt} Ладонкина М.Е. \hfill}\\ \\ 
    \hspace*{1cm}\hspace*{-1cm}Заведующий кафедрой \hfill\\
    \hspace*{1cm}\hspace*{-1cm}к.ф.-м.н.\hfill\hbox to\maxl{\rule{5em}{1pt} Муравьев С.Е. \hfill}\\ \\ 
    
    }
    
   
    
    


\end{titlepage}


    % Аннотация
    \section*{Аннотация}
    В работе проведён подробный анализ численного метода разностных схем решения уравнения теплопроводности, исследуются преимущества и недостатки использования равномерных статических сеток в случае квазилинейных уравнений с существенно различающимися масштабами (как пространственными, так и временными) физическими процессами.
Разработано программное обеспечение, реализующее рассмотренные схемы численного решения, проверены описанные особенности каждой и схем.
Показана необходимость использования сеток с локальным измельчением, подстраивающееся под особенности решения.
Описаны некоторые теоретические основы данного метода.
Реализован в виде программного кода соответствующий алгоритм, работоспособность которого проверена на нескольких тестовых задачах.
Показаны преимущества использования блочных локально-адаптивных сеток по сравнению с использованием равномерных статических.
    \newpage

    % Содержание
    \tableofcontents
    \newpage

    % Введение
    \section{Введение}
    Здесь будет введение

    % Основные понятия теории разностных схем
    \section{Основные понятия теории разностных схем}\label{sec:MainDiffSchemes}
    Математическая формулировка физических задач, описанных во введении имеет вид:
\begin{equation}\label{eq:InitialProblem}
    \begin{cases}
        L[u](x) = f(x), & x = (x_1, \ldots, x_n)^{T} \in G \subset \mathbb{R}^n\\
        \Gamma[u](x) = \mu(x), & x \in \partial G
    \end{cases}, 
\end{equation}
где 
\begin{itemize}
    \item $L$~--- дифференциальный оператор уравнения;
    \item $\Gamma$~--- оператор начально-краевых условий (в общем случае также дифференциальный);
    \item $f$, $\mu$~--- заданные функции.
\end{itemize}
Решения исходной задачи~--- функции $u(x)$ непрерывного аргумента $x \in G$, являются элементами некоторого функционального пространства $H_0$ с нормой $\norm{\cdot}$.
В методе конечных разностей область $G$ заменяется на некоторое дискретное множество точек $\omega_h$, именуемое \emph{сеткой}, а функциональное пространство $H_0$ заменяется на $H_h$~--- гильбертово пространство сеточных функций ${y_h : G \supset \omega_h \rightarrow \mathbb{R}}$, где $h$~--- некоторый параметр, характеризующий сетку $\omega_h$ в области $G$.
Например, равномерная статическая сетка:
\begin{equation*}
    \omega_h = \{x = (x_1, \ldots, x_n) \in G \mid x_i = h_i \cdot k,\,\, k = 0, 1, \ldots, N_i\,\, i = 1, \ldots, n\}
\end{equation*}
Получив приближённое решение задачи $y_h$, необходимо оценивать степень \glqq близости\grqq к решению исходной задачи $u(x)$.
$y_h$~и~$u$ являются элементами разных функциональных пространств, поэтому для оценивания близости в работе используется проекционный метод: пространство $H_0$ отображается (проектируется) на пространство $H_h$ оператором $\mathcal{P}$:
\begin{equation*}
    \mathcal{P}_h \colon H_0 \ni u \mapsto u_h \in H_h.
\end{equation*}
Простейший выбор: ограничение $u$ на сетку $\omega_h$:
\begin{equation*}
    u_h(x) := u(x),\quad x \in \omega_h \subset G
\end{equation*}
Иногда пользуются \glqq более равномерным\grqq способом ограничения с усреднением по окрестности узла:
\begin{equation*}
    u_h(x) := \frac{1}{2h} \int_{x - h}^{x + h} u(x') dx'
\end{equation*}
Тогда близость приближённого решения $y_h$ и исходного решения $u$ оценивается по норме $\norm{\cdot}_{h}$ пространства $H_h$:
\begin{equation*}
    e = \norm{y_h - u_h}_{h},
\end{equation*}
при этом требуется, чтобы она аппроксимировала норму $\norm{\cdot}_{0}$ в слудеющем смысле\cite{СамарскийТеорияРазностныхСхем}:
\begin{equation*}
    \lim\limits_{h \rightarrow 0} \norm{u_h}_h = \norm{u}_{0}
\end{equation*}
В работе используется норма
$
    \norm{y}_{h} = \sqrt{
        \sum_{i=1}^{N} hy_i^2
    }
$.
Исходному дифференциальному оператору $L$ ставится в соответствие \emph{разностный оператор} $L_h$:
\begin{equation*}
    L_h[v] (x) = \sum_{x' \in T(x)} A_h(x, x') v(x'),
\end{equation*}
где $T(x)$~--- некоторое множество узлов сетки, называемое \emph{шаблоном}.
Например, двумерный оператор Лапласа $L = \Delta$ на двумерной равномерной сетке можно аппроксимировать, используя шаблон \glqq крест\grqq (\seefigref{fig:cross}):
\begin{equation*}
    \Delta u(x) = \frac{\partial^2 u}{\partial x_1^2} + \frac{\partial^2 u}{\partial x_2^2} \mapsto 
    \frac{u_{i+1}^{j} - 2u_{i}^{j} + u_{i-1}^j}{h_1^2} +
    \frac{u_i^{j + 1} - 2u_i^{j} + u_i^{j-1}}{h_2^2}, 
\end{equation*}
где $u_i^j = u({x_1}_i, {x_2}_j),\quad ({x_1}_i, {x_2}_j) \in \omega_h$.
\begin{figure}
    \centering
    \begin{tikzpicture}
        \filldraw[black] (0, 0) circle(2pt) node[anchor=south west] {$(x_1, x_2)$};
        \draw (-2.5, 0) node[anchor=north] {$(x_1 - h_1, x_2)$} -- (2.5, 0) node[anchor=north] {$(x_1 + h_1, x_2)$};
        \filldraw[black] (-2.5, 0) circle(2pt);
        \filldraw[black] (2.5, 0) circle(2pt);
        \filldraw[black] (0, -2.5) circle(2pt);
        \filldraw[black] (0, 2.5) circle(2pt);
        \draw (0, -2.5) node[anchor=north] {$(x_1, x_2 - h_2)$} -- (0, 2.5) node[anchor=south] {$(x_1, x_2 + h_2)$};
    \end{tikzpicture}
    \caption{Шаблон \glqq Крест\grqq}
    \label{fig:cross}
\end{figure}
Погрешность аппроксимации оператора $L$ разностным оператором $L_h$ определяется как сеточная функция $\psi_h = L_h[u_h] - (L[u])_h, u \in H_0$.
Если $\norm{\psi_h} = O(|h|^m)$, то говорят, что оператор $L_h$ аппроксимирует оператор $L$ с порядком $m$.
Если $\psi(x) = O(h^m), m$, то говорят, что оператор $L_h$ аппроксимирует оператор $L$ в точке $x$ с порядком $m$.

Теперь сформулируем непосредственно то, что называется \emph{разностной схемой}, алгоритмы решения которой и реализуются на компьютере.
Исходной задаче \eqref{eq:InitialProblem} ставится в соответствие семейство разностных задач, зависящих от параметра $h$, называемое разностной схемой:
\begin{equation*}
    \left\{ 
        \begin{cases}
            L_h [y_h] = \phi_h, & x \in \omega_h\\
            l_h [y_h] = \chi_h, & x \in \gamma_h
        \end{cases}
     \right\}_h,\quad \phi_h = \mathcal{P}_h[f], \chi_h = \mathcal{P}_h[\mu]
\end{equation*}
Под погрешностью разностной схемы понимается $z_h = y_h - u_h$, где $u_h = \mathcal{P}_h u$~--- проекция решения исходной задачи на $H_0$. 
Решения разностной задачи сходится к решению исходной задачи, если 
\begin{equation*}
    \norm{z_h}_{h} \rightarrow 0 \text{ при } |h| \rightarrow 0
\end{equation*}
Введём также понятие устойчивости схемы.
Разностная схема называется устойчивой (корректной, сходящейся), если
$
    \exists h_0 > 0 : \forall h(|h| \le h_0) \Rightarrow
$
\begin{equation*}
    \begin{aligned}
        &1. \forall \phi \in H_h \quad \exists! y_h\text{--- решение};\\
        &2. \exists M > 0 : \forall \phi_h, \tilde{\phi}_h \norm{y_h - \tilde{y}_h} \le M \norm{\phi_h - \tilde{\phi}_h}
    \end{aligned}
\end{equation*}

На этом только лишь математическая сторона вопроса формулирования проблемы завершена. 
Дальнейшие шаги по исследованию разностной схемы опираются на конкретный выбор множества сеток $\left\{ \omega_h \right\}$ и аппроксимирующего оператора $L_h$, выбор которого, в свою очередь, существенно зависит от некоторых вопросов реализации получаемого алгоритма на компьютере.







% например, в случае квазилинейного уравнения теплопроводности
% $
%     L = \frac{\partial }{\partial t} - 
%     \sum\limits_{i = 1}^{n} \frac{\partial }{\partial x_i} \left( 
%         k(x) \frac{\partial u}{\partial x_i}
%      \right)
% $;

% Сетки и сеточные функции
% Пусть $H_0$~--- некоторое функциональное пространство функций $u(x)$ непрерывного аргумента $x \in G$ с нормой $\|\cdot\|_0$.
% В методе конечных разностей область $\bar{G}$ изменения аргумента $x$ заменяется сеткой $\bar{\omega}_h$~--- конечным множеством точек $x_i \in \bar{G}$,
% а функциональное пространство $H_0$ заменяется на $H_h$~--- гильбертово пространство сеточных функций $y_h(x)$, определённых на сетке $\bar{\omega}_h$ с нормой $\|\cdot\|_h$.
% Обычно будем пользоваться нормой $\|y\|_h = \left( \sum\limits_{i = 1}^{N} h y_i^2 \right)^{1/2}$
% Функции $y_h(x) \in H_h$~--- численные решения, аппроксимации исходных решений $u(x) \in H_0$.
% Соответственно, основной интерес теории приближённых методов представляет \textbf{оценка близости} $y_h$ к $u$.
% Эти два вектора являются элементами разных пространств. Их близость описываем следующим образом:

% $\mathcal{P}_h: H_0 \rightarrow H_h,\quad u \mapsto u_h: u_h(x) = u(x), x \in \bar{\omega}_h$.
% Тогда близость $u_h$ и $y_h$ характеризуется числом $\|y_h - u_h\|_h$.

% Условие согласования норм в $H_h$ и $H_0$:
% $\lim\limits_{h \rightarrow 0} \|u_h\|_h = \| u\|_0$, или другими словами "норма $\|\cdot\|_h$ аппроксимирует норму $\|\cdot\|_0$"


% Разностная аппроксимация дифференциальных операторов

% Пусть $L$~--- линейный дифференциальный оператор, $\mathcal{D}(L) = H_0$, $\quad Ш(x) \subset \bar{\omega}_h$~--- некоторое множество узлов сетки

% \textbf{Опр.} $L_h v_h (x) = L_h v_h(x) = \sum\limits_{\xi \in Ш(x)} A_h(x, \xi) v_n (\xi)$~--- разностный оператор, разностная аппроксимация оператора $L$

% \textbf{Опр.} $\psi(x) = L_h v(x) - L v(x)$~--- локальная погрешность разностной аппроксимации $Lv$ в точке $x$.

% \textbf{Опр.} $L_h$ аппроксимирует $L$ с порядком $ m > 0$ в точке $x$, если $\psi(x) = L_h v(x) - Lv(x) = O(h^m)$

% \textbf{Опр.} Погрешность аппроксимации оператора $L$ разностным оператором $L_h$~--- это сеточная функция
% $\psi_h = L_h v_h - (Lv)_h, \quad (Lv)_h = \mathcal{P}_h (Lv), \quad v \in H_0$

% \textbf{Опр.} Разностный оператор $L_h$ аппроксимирует дифференциальный оператор $L$ с порядком $m>0$, если
% $$
%     \| \psi_h \| = \| L_h v_h - (Lv)_h \|_h = O(|h|^m)
% $$

% Общее утверждение, касательно аппроксимации дифференциальных операторов разностными таково, что:
% * погрешность аппроксимации зависит от используемого шаблона, причём можно достичь любого порядка локальной аппроксимации повышением кол-ва узлов шаблона (однако ухудшается качество операторов)
% * исследование локальной аппроксимации может оказаться недостаточным для суждения о порядке разностной аппроксимации на сетке и тем самым для суждения о качестве разностного оператора
% * рассмотрение разностной аппроксимации на решении дифференциального уравнения может использоваться для повышения порядка аппроксимации


% Постановка разностных задач.

% $G\subset \mathbb{R}^n$, $\partial G = \Gamma$, $L, l$~--- линейные дифференциальные операторы, $\mathcal{D}(L) = H_0$, $\mathcal{D}(l) = H_0$. 
% Исохдной задаче
% $$
%     \begin{cases}
%         Lu(x) = f(x), & x \in G\\
%         lu(x) = \mu (x), & x \in \Gamma
%     \end{cases}
% $$
% ставится в соответствие *семейство разностных задач*, зависящих от параметра $h$, называемое *разностной схемой*:
% $$
%     \left\{ 
%         \begin{cases}
%             L_h y_h = \varphi_h, & x \in \omega_h\\
%             l_h y_h = \chi_h, & x \in \gamma_h
%         \end{cases}
%         \right\}_h
% $$

% \textbf{Опр.} Погрешность разностной схемы $z_h = y_h - u_h$, где $y_h$~--- решение разностной задачи, а $u_h = \mathcal{P}_h u$ проекция решения исходной задачи на $H_0$.

% \textbf{Опр.} Погрешность аппроксимации для уравнения $L_h y_h = \varphi_h$ на решении $u(x)$ уравнения $Lu = f$:
% $$
%     \psi_h = L_h z_h = \varphi_h - L_h u_h
% $$
% погрешность аппроксимации для условия $l_h y_h = \chi_h$ на решении $u(x)$ исходной задачи:
% $$
%     \nu_h = l_h z_h = \chi_h - l_h u_h
% $$

% \textbf{Опр.} Решение разностной задачи *сходится к решению исходной задачи*, если 
% $$
%     \| z_h\|_h = \|y_h - u_h \|_h \rightarrow 0 \quad  |h| \rightarrow 0
% $$

% \textbf{Опр.} Разностная схема *сходится со скоростью* $O(|h|^m)$ (*имеет $m$-ый порядок точности*), если 
% $$
%     \| z_h \|_h = \| y_h - u_h \|_h = O(|h|^m), \quad |h| \rightarrow 0
% $$

% \textbf{Опр.} Разностная схема обладает $n$-ым порядком аппроксимации, если
% $$
%     \| \psi_h \|_h = O(|h|^n),\quad \| \nu_h \|_h = O(|h|^n)
% $$

% \textbf{Опр.} Разностная схема называется корректной (устойчивой, сходящейся), если
% $\exists h_0 > 0 : \forall h (|h| \leqslant h_0) \Rightarrow$
% 1. $\forall \varphi \in \tilde{H}_h \,\, \exists ! y_h$~--- решение;
% 2. $\exists M > 0 : \forall \varphi_h, \tilde{\varphi}_h \quad \| y_h - \tilde{y}_h \| \leqslant M \| \varphi_h - \tilde{\varphi}_h \|$

% **Утв.** Если схема устойчива и аппроксимирует исходную, т.е. $\| \psi_h \|_h = O(|h|^n)$, то она сходится, причём порядок сходимости совпадает с порядком аппроксимации.

    % Разностные схемы на статических сетках
    \section{Разностные схемы на статических сетках}\label{sec:StaticGrid}
        % Общие замечания
        % Избранные разностные схемы для ур-ния теплопроводности
        \subsection{Избранные разностные схемы для уравнения теплопроводности}
            % Явная схема
            \subsubsection{Явная схема}
            % Неявные симметричные схемы
            \subsubsection{Однопараметрическое семейство неявных схем}
            % Локально-одномерные схемы
            \subsubsection{Локально-одномерные схемы}
        % Программный код, примеры расчётов
        \subsection{Программный код, примеры расчётов}
    
    % Теория блочных локально-адаптивных сеток
    \section{Теория блочных локально-адаптивных сеток}\label{sec:LAG}
    
    % Замечаения о программной реализации
    \section{Замечания о программной реализации}

    % Основные результаты
    \section{Основные результаты}

    % Приложения к физике, реальные задачи
    \section{Приложения к физике, реальные задачи}

    % Заключение
    \section{Заключение}

    % Список Литературы
    \newpage
    \renewcommand{\refname}{Список литературы}
    \printbibliography
    \addcontentsline{toc}{section}{\refname}
\end{document}
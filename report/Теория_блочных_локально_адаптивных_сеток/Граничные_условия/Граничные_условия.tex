Поступают следующим образом: при решение на уровне $l$, в качестве граничных значений для обновления $t_j \mapsto t_{j + 1}$ берутся уже найденные приблежённые значения на слое $t_{j + 1}$ на уровне $l - 1$ (то есть на предыдущем уровне).
Значение в точках $\partial \omega_l \cup \omega_{l - 1} \not\subset \omega_{l + 1}$ задаётся путём линейной интерполяции со значений в точках $\partial \omega_l \cup \omega_{l - 1} \subset \omega_{l + 1}$.
Рисунок \ref{fig:linear_interpolation} описывает вышескзаанное:
так, точки $A$, $B$ принадлежат уровню $l - 1$.
Точки $C$, $D$, $E$ принаждлежат уровню $l$.
Точки $C$ и $E$ совпадают с точками $A$ и $B$, и решение в них задаётся с уровня $l - 1$, а значение в \glqq промежуточной\grqq точке $D$ задаётся линейной интерполяцией значений с точек $A$ и $B$:
\begin{equation*}
    u_l(D) = \frac{u_{l - 1}(A) + u_{l - 1}(B)}{2}
\end{equation*}
Для большей точности расчёта помимо измелчение пространственной сетки происходит и измельчение временной сетке, а именно:
\begin{equation*}
    \frac{\Delta x_{l}}{\Delta x_{l - 1}} = \frac{\Delta t_{l}}{\Delta t_{l - 1}}.
\end{equation*}
Это позволяет, например, избежать появлений неустойчивости в случае использования явной схемы
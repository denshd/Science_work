Поступают следующим образом: при решение на уровне $l$, в качестве граничных значений для обновления $t_j \mapsto t_{j + 1}$ берутся уже найденные приблежённые значения на слое $t_{j + 1}$ на уровне $l - 1$ (то есть на предыдущем уровне).
Значение в точках $\partial \omega_l \cup \omega_{l - 1} \not\subset \omega_{l + 1}$ задаётся путём линейной интерполяции со значений в точках $\partial \omega_l \cup \omega_{l - 1} \subset \omega_{l + 1}$.
Рисунок \ref{fig:linear_interpolation} описывает вышескзаанное.
\begin{figure}[h]
    \centering
    \begin{tikzpicture}
        \filldraw (0, 0) circle (2pt);
        \filldraw(2, 0) circle (2pt);
        \filldraw(3, 0) circle (2pt);
        \filldraw(4, 0) circle (2pt);
        \filldraw(6, 0) circle (2pt);

        \filldraw (0, 2) circle (2pt);
        \filldraw(2, 2) circle (2pt) node[anchor=south east] {$A$};
        \filldraw(3, 2) circle (2pt) node[anchor=north] {$D$};
        \filldraw(4, 2) circle (2pt) node[anchor=south west] {$B$};
        \filldraw(6, 2) circle (2pt);

        \filldraw(2, 1) circle (2pt);
        \filldraw(3, 1) circle (2pt);
        \filldraw(4, 1) circle (2pt);

        \draw (2, 2) arc (150:30:1.15);

        \draw [-{Stealth[length=3mm, width=2mm]}] (3, 3.2) node[anchor=west]{$u_l(D) =\frac{1}{2}(u_{l - 1}(A) + u_{l - 1}(B))$} -- (3, 2.2);
    \end{tikzpicture}
    \caption{Интерполяция с грубой сетки}
    \label{fig:linear_interpolation}
\end{figure}
Так, точки $A$, $B$ принадлежат уровню $l - 1$.
Точки $C$, $D$, $E$ принаждлежат уровню $l$.
Точки $C$ и $E$ совпадают с точками $A$ и $B$, и решение в них задаётся с уровня $l - 1$, а значение в \glqq промежуточной\grqq точке $D$ задаётся линейной интерполяцией значений с точек $A$ и $B$:
\begin{equation*}
    u_l(D) = \frac{u_{l - 1}(A) + u_{l - 1}(B)}{2}
\end{equation*}
Для большей точности расчёта помимо измелчение пространственной сетки происходит и измельчение временной сетке, а именно:
\begin{equation*}
    \frac{\Delta x_{l}}{\Delta x_{l - 1}} = \frac{\Delta t_{l}}{\Delta t_{l - 1}}.
\end{equation*}
Это позволяет, например, избежать появлений неустойчивости в случае использования явной схемы.
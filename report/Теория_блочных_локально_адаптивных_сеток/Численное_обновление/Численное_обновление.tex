Пускай известно решение на временном слое $j$, то есть известно $u_{i, k}^{j} \quad \forall i, k$.
Решение на новом временном слое ищется постепенно, от груого к мелкому, используя какую-либо ранее расмотренную разностную схему для статических сеток.
Как уже отмечалось, использование явной схемы нецелесообразно ввиду отсутствия устойчивости.
В работах \cite{жуков2015численное} используется явно-итерационная схема ЛИ-М, опирающаяся на оптимальные свойства многочлена Чебышева специальной конструкции, о чём подробнее можно прочитать в \cite{жуков2010явных, жуков2015решении, жуков2014параллельный}
Предпочтение было отдано локально-одномерной схеме, ввиду всех её преимуществ, описанных ранее в разделе \ref{sec:LOS}

Сначала обновляется решение на самой крупной сетке, то есть на уровне~$l$.
Затем ищется решение на более точной сетке. 
Так как в общем случае $\partial G_l \not\subset \partial G_1$, то сразу не понятно, как учитывать граничные условия в схеме обновления решения.
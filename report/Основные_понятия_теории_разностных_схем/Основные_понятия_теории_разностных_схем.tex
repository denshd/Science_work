Математическая формулировка физических задач, описанных во введении имеет вид:
\begin{equation}\label{eq:InitialProblem}
    \begin{cases}
        L[u](x) = f(x), & x = (x_1, \ldots, x_n)^{T} \in G \subset \mathbb{R}^n\\
        \Gamma[u](x) = \mu(x), & x \in \partial G
    \end{cases}, 
\end{equation}
где 
\begin{itemize}
    \item $L$~--- дифференциальный оператор уравнения;
    \item $\Gamma$~--- оператор начально-краевых условий (в общем случае также дифференциальный);
    \item $f$, $\mu$~--- заданные функции,
    \item $G$~--- заданная область из $\mathbb{R}^n$, $\partial G$~--- её граница.
\end{itemize}
Решения исходной задачи~--- функции $u(x)$ непрерывного аргумента $x \in G$, являются элементами некоторого функционального пространства $H_0$ с нормой $\norm{\cdot}$.
В методе конечных разностей область $G$ заменяется на некоторое дискретное множество точек $\omega_h$, именуемое \emph{сеткой}, а функциональное пространство $H_0$ заменяется на $H_h$~--- гильбертово пространство сеточных функций ${y_h : G \supset \omega_h \rightarrow \mathbb{R}}$, где $h$~--- некоторый параметр, характеризующий сетку $\omega_h$ в области $G$.
Например, равномерная статическая сетка:
\begin{equation*}
    \omega_h = \{x = (x_1, \ldots, x_n) \in G \mid x_i = h_i \cdot k,\,\, k = 0, 1, \ldots, N_i\,\, i = 1, \ldots, n\}
\end{equation*}
Получив приближённое решение задачи $y_h$, необходимо оценивать степень \glqq близости\grqq к решению исходной задачи $u(x)$.
$y_h$~и~$u$ являются элементами разных функциональных пространств, поэтому для оценивания близости в работе используется проекционный метод: пространство $H_0$ отображается (проектируется) на пространство $H_h$ оператором $\mathcal{P}$:
\begin{equation*}
    \mathcal{P}_h \colon H_0 \ni u \mapsto u_h \in H_h.
\end{equation*}
Простейший выбор: ограничение $u$ на сетку $\omega_h$:
\begin{equation*}
    u_h(x) := u(x),\quad x \in \omega_h \subset G
\end{equation*}
Иногда пользуются \glqq более равномерным\grqq способом ограничения с усреднением по окрестности узла:
\begin{equation*}
    u_h(x) := \frac{1}{2h} \int_{x - h}^{x + h} u(x') dx'
\end{equation*}
Тогда близость приближённого решения $y_h$ и исходного решения $u$ оценивается по норме $\norm{\cdot}_{h}$ пространства $H_h$:
\begin{equation*}
    e = \norm{y_h - u_h}_{h},
\end{equation*}
при этом требуется, чтобы она аппроксимировала норму $\norm{\cdot}_{0}$ в слудеющем смысле\cite{СамарскийТеорияРазностныхСхем}:
\begin{equation*}
    \lim\limits_{h \rightarrow 0} \norm{u_h}_h = \norm{u}_{0}
\end{equation*}
В работе используется норма
$
    \norm{y}_{h} = \sqrt{
        \sum_{i=1}^{N} hy_i^2
    }
$.

Исходному дифференциальному оператору $L$ ставится в соответствие \emph{разностный оператор} $L_h$:
\begin{equation*}
    L_h[v] (x) = \sum_{x' \in T(x)} A_h(x, x') v(x'),
\end{equation*}
где $T(x)$~--- некоторое множество узлов сетки, называемое \emph{шаблоном}, $A_h(x, x')$~--- некоторые коэффициенты.
Например, двумерный оператор Лапласа $L = \Delta$ на двумерной равномерной сетке можно аппроксимировать, используя шаблон \glqq крест\grqq (\seefigref{fig:cross}):
\begin{equation*}
    \Delta u(x) = \frac{\partial^2 u}{\partial x_1^2} + \frac{\partial^2 u}{\partial x_2^2} \mapsto 
    \frac{u_{i+1}^{j} - 2u_{i}^{j} + u_{i-1}^j}{h_1^2} +
    \frac{u_i^{j + 1} - 2u_i^{j} + u_i^{j-1}}{h_2^2}, 
\end{equation*}
где $u_i^j = u({x_1}_i, {x_2}_j),\quad ({x_1}_i, {x_2}_j) \in \omega_h$.
\begin{figure}
    \centering
    \begin{tikzpicture}
        \filldraw[black] (0, 0) circle(2pt) node[anchor=south west] {$(x_1, x_2)$};
        \draw (-2.5, 0) node[anchor=north] {$(x_1 - h_1, x_2)$} -- (2.5, 0) node[anchor=north] {$(x_1 + h_1, x_2)$};
        \filldraw[black] (-2.5, 0) circle(2pt);
        \filldraw[black] (2.5, 0) circle(2pt);
        \filldraw[black] (0, -2.5) circle(2pt);
        \filldraw[black] (0, 2.5) circle(2pt);
        \draw (0, -2.5) node[anchor=north] {$(x_1, x_2 - h_2)$} -- (0, 2.5) node[anchor=south] {$(x_1, x_2 + h_2)$};
    \end{tikzpicture}
    \caption{Шаблон \glqq Крест\grqq}
    \label{fig:cross}
\end{figure}
Погрешность аппроксимации оператора $L$ разностным оператором $L_h$ определяется как сеточная функция $\psi_h = L_h[u_h] - (L[u])_h, u \in H_0$.
Если $\norm{\psi_h} = O(|h|^m)$, то говорят, что оператор $L_h$ аппроксимирует оператор $L$ с порядком $m$.
Если $\psi(x) = O(h^m), m$, то говорят, что оператор $L_h$ аппроксимирует оператор $L$ в точке $x$ с порядком $m$.

Теперь сформулируем непосредственно то, что называется \emph{разностной схемой}, алгоритмы решения которой и реализуются на компьютере.
Исходной задаче \eqref{eq:InitialProblem} ставится в соответствие семейство разностных задач, зависящих от параметра $h$, называемое разностной схемой:
\begin{equation*}
    \left\{ 
        \begin{cases}
            L_h [y_h] = \phi_h, & x \in \omega_h\\
            l_h [y_h] = \chi_h, & x \in \gamma_h
        \end{cases}
     \right\}_h,\quad \phi_h = \mathcal{P}_h[f], \chi_h = \mathcal{P}_h[\mu]
\end{equation*}
Под погрешностью разностной схемы понимается $z_h = y_h - u_h$, где $u_h = \mathcal{P}_h u$~--- проекция решения исходной задачи на $H_0$. 
Решение разностной задачи сходится к решению исходной задачи, если 
\begin{equation*}
    \norm{z_h}_{h} \rightarrow 0 \text{ при } |h| \rightarrow 0
\end{equation*}
Введём также понятие устойчивости схемы.
Разностная схема называется устойчивой (корректной, сходящейся), если
$
    \exists h_0 > 0 : \forall h(|h| \le h_0) \Rightarrow
$
\begin{equation*}
    \begin{aligned}
        &1. \forall \phi \in H_h \quad \exists! y_h\text{--- решение};\\
        &2. \exists M > 0 : \forall \phi_h, \tilde{\phi}_h \norm{y_h - \tilde{y}_h} \le M \norm{\phi_h - \tilde{\phi}_h}
    \end{aligned}
\end{equation*}

На этом только лишь математическая сторона вопроса формулирования проблемы завершена. 
Дальнейшие шаги по исследованию разностной схемы опираются на конкретный выбор множества сеток $\left\{ \omega_h \right\}$ и аппроксимирующего оператора $L_h$, выбор которого, в свою очередь, существенно зависит от некоторых вопросов реализации получаемого алгоритма на компьютере.
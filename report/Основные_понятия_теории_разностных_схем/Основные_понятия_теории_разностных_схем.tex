\textbf{Ну, тут я пока просто скопировал то, что было написано ранее для себя, всё это будет переформулировываться в более красивой форме, и, конечно же, добавится вода.
Формулы тоже нужно переписать, всё было написано и тупо скопировано пока из markdown'а}
    
Основные положения метода конечных разностей

Сетки и сеточные функции
Пусть $H_0$~--- некоторое функциональное пространство функций $u(x)$ непрерывного аргумента $x \in G$ с нормой $\|\cdot\|_0$.
В методе конечных разностей область $\bar{G}$ изменения аргумента $x$ заменяется сеткой $\bar{\omega}_h$~--- конечным множеством точек $x_i \in \bar{G}$,
а функциональное пространство $H_0$ заменяется на $H_h$~--- гильбертово пространство сеточных функций $y_h(x)$, определённых на сетке $\bar{\omega}_h$ с нормой $\|\cdot\|_h$.
Обычно будем пользоваться нормой $\|y\|_h = \left( \sum\limits_{i = 1}^{N} h y_i^2 \right)^{1/2}$
Функции $y_h(x) \in H_h$~--- численные решения, аппроксимации исходных решений $u(x) \in H_0$.
Соответственно, основной интерес теории приближённых методов представляет \textbf{оценка близости} $y_h$ к $u$.
Эти два вектора являются элементами разных пространств. Их близость описываем следующим образом:

$\mathcal{P}_h: H_0 \rightarrow H_h,\quad u \mapsto u_h: u_h(x) = u(x), x \in \bar{\omega}_h$.
Тогда близость $u_h$ и $y_h$ характеризуется числом $\|y_h - u_h\|_h$.

Условие согласования норм в $H_h$ и $H_0$:
$\lim\limits_{h \rightarrow 0} \|u_h\|_h = \| u\|_0$, или другими словами "норма $\|\cdot\|_h$ аппроксимирует норму $\|\cdot\|_0$"


Разностная аппроксимация дифференциальных операторов

Пусть $L$~--- линейный дифференциальный оператор, $\mathcal{D}(L) = H_0$, $\quad Ш(x) \subset \bar{\omega}_h$~--- некоторое множество узлов сетки

\textbf{Опр.} $L_h v_h (x) = L_h v_h(x) = \sum\limits_{\xi \in Ш(x)} A_h(x, \xi) v_n (\xi)$~--- разностный оператор, разностная аппроксимация оператора $L$

\textbf{Опр.} $\psi(x) = L_h v(x) - L v(x)$~--- локальная погрешность разностной аппроксимации $Lv$ в точке $x$.

\textbf{Опр.} $L_h$ аппроксимирует $L$ с порядком $ m > 0$ в точке $x$, если $\psi(x) = L_h v(x) - Lv(x) = O(h^m)$

\textbf{Опр.} Погрешность аппроксимации оператора $L$ разностным оператором $L_h$~--- это сеточная функция
$\psi_h = L_h v_h - (Lv)_h, \quad (Lv)_h = \mathcal{P}_h (Lv), \quad v \in H_0$

\textbf{Опр.} Разностный оператор $L_h$ аппроксимирует дифференциальный оператор $L$ с порядком $m>0$, если
$$
    \| \psi_h \| = \| L_h v_h - (Lv)_h \|_h = O(|h|^m)
$$

Общее утверждение, касательно аппроксимации дифференциальных операторов разностными таково, что:
* погрешность аппроксимации зависит от используемого шаблона, причём можно достичь любого порядка локальной аппроксимации повышением кол-ва узлов шаблона (однако ухудшается качество операторов)
* исследование локальной аппроксимации может оказаться недостаточным для суждения о порядке разностной аппроксимации на сетке и тем самым для суждения о качестве разностного оператора
* рассмотрение разностной аппроксимации на решении дифференциального уравнения может использоваться для повышения порядка аппроксимации


Постановка разностных задач.

$G\subset \mathbb{R}^n$, $\partial G = \Gamma$, $L, l$~--- линейные дифференциальные операторы, $\mathcal{D}(L) = H_0$, $\mathcal{D}(l) = H_0$. 
Исохдной задаче
$$
    \begin{cases}
        Lu(x) = f(x), & x \in G\\
        lu(x) = \mu (x), & x \in \Gamma
    \end{cases}
$$
ставится в соответствие *семейство разностных задач*, зависящих от параметра $h$, называемое *разностной схемой*:
$$
    \left\{ 
        \begin{cases}
            L_h y_h = \varphi_h, & x \in \omega_h\\
            l_h y_h = \chi_h, & x \in \gamma_h
        \end{cases}
        \right\}_h
$$

\textbf{Опр.} Погрешность разностной схемы $z_h = y_h - u_h$, где $y_h$~--- решение разностной задачи, а $u_h = \mathcal{P}_h u$ проекция решения исходной задачи на $H_0$.

\textbf{Опр.} Погрешность аппроксимации для уравнения $L_h y_h = \varphi_h$ на решении $u(x)$ уравнения $Lu = f$:
$$
    \psi_h = L_h z_h = \varphi_h - L_h u_h
$$
погрешность аппроксимации для условия $l_h y_h = \chi_h$ на решении $u(x)$ исходной задачи:
$$
    \nu_h = l_h z_h = \chi_h - l_h u_h
$$

\textbf{Опр.} Решение разностной задачи *сходится к решению исходной задачи*, если 
$$
    \| z_h\|_h = \|y_h - u_h \|_h \rightarrow 0 \quad  |h| \rightarrow 0
$$

\textbf{Опр.} Разностная схема *сходится со скоростью* $O(|h|^m)$ (*имеет $m$-ый порядок точности*), если 
$$
    \| z_h \|_h = \| y_h - u_h \|_h = O(|h|^m), \quad |h| \rightarrow 0
$$

\textbf{Опр.} Разностная схема обладает $n$-ым порядком аппроксимации, если
$$
    \| \psi_h \|_h = O(|h|^n),\quad \| \nu_h \|_h = O(|h|^n)
$$

\textbf{Опр.} Разностная схема называется корректной (устойчивой, сходящейся), если
$\exists h_0 > 0 : \forall h (|h| \leqslant h_0) \Rightarrow$
1. $\forall \varphi \in \tilde{H}_h \,\, \exists ! y_h$~--- решение;
2. $\exists M > 0 : \forall \varphi_h, \tilde{\varphi}_h \quad \| y_h - \tilde{y}_h \| \leqslant M \| \varphi_h - \tilde{\varphi}_h \|$

**Утв.** Если схема устойчива и аппроксимирует исходную, т.е. $\| \psi_h \|_h = O(|h|^n)$, то она сходится, причём порядок сходимости совпадает с порядком аппроксимации.
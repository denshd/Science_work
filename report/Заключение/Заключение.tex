В работе был проведён подробный анализ некоторых разностных схем численного решения уравнения теплопроводности, исследование их устойчивости, точности и скорости счёта, а именно:
\begin{itemize}
    \item явной разностной схемы
    \item однопараметрического семейства неявных разностных схем, в том числе чисто неявной схемы для квазилинейного уравнения
    \item локально-одномерной схемы
\end{itemize}
Проделанный анализ сопровождался проведением большого количества численных расчётов с помощью написаного программного обеспечения, реализующего все вышеперечисленные разностные схемы и позволяющего решать задачу Дирихле для одномерного и многомерного уравнения теплопроводности с произвольными граничными и начальными данными, а также позволяющего демонстрировать точность и скорость счёта разностных схем.
Показаны преимущества и недостатки использования равномерных статических сеток на конкретных примерах из различных статей.

Сделан обзор литературы, посвещённой методу блочно-структурированных локально-адаптивных сеток (block-structured adaptive mesh refinement), разработан алгоритм решения квазилинейного уравнения теплопроводности на блочно-структурированной сетке с использованием локально-одномерной схемы для обеспечения устойчивости метода, который реализован в виде программного кода.
Исследована работоспособность алгоритма, показаны преимущества использования локального измельчения сетки.

Разработка численных методов локальной-адаптации для решения прикладных задач является актуальной и неослабевающей научной проблемой, привлекающей как исследователей из области вычислительной математики, так и физиков-экспериментаторов благодаря возможности поставить дорогостоящие эксперименты виртуально на компьютере.
В настоящее время проблема полностью не решена и в научном мире ведутся активные исследования на эту тему.

В будущем планируется разработка программного обеспечения с переносом функционала, реализованного в \cite{ranocha2022adaptive, schlottkelakemper2021purely, schlottkelakemper2020trixi} для гиперболических проблем, на случай параболических уравнений, областей произвольной формы и возможности распараллеливания программы.
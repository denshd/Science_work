Большинство моделей классической физики, таких как гидрогазодинамика, описываются начально-краевыми задачами для дифференциальных уравнений в частных производных второго порядка\cite{ТихоновСамарский, ЛандауГидродинамика}.
Нахождение аналитического решения таких задач в общем случае не представляется возможным.
Основной метод численного решения таких задач (то есть представление как исходных данных в задаче, так и её решения в конечном итоге~--- в виде множества чисел)~--- метод разностных схем, который подробно описан в разделе~\ref{sec:MainDiffSchemes}.
Его основа заключается в том, что исходная непрерывная задача в области $G \subset \mathbb{R}^n$ сводится к \emph{семейству разностных задач}~--- системам конечного числа линейных (в общем случае~--- нелинейных) уравнений, которые решаются алгоритмически и тем самым могут быть программно реализованы на современных ЭВМ.

Принципиальная возможность применения тех или иных алгоритмов основывается на вопросе об их сходимости, точности и устойчивости.
Так, в работе \cite{СамарскийТеорияРазностныхСхем} даётся обширное описание алгоритмов решения задач на \emph{статических} сетках, исследуются вопросы устойчивости и скорости сходимости.
Более подробное исследование тех же вопросов в случае \emph{неравномерных статических сеток} дан в работе \cite{СамарскийНеравномерныеСетки}.

Проблема использования таких сеток \ldots

\texttt{
    Здесь написать про то, что в квазилинейном уравненнии не существует классического решения, обобщённые решения, разрывы, волновые фронты, ударные волны, все дела. В статьях Самарского там показано это, тыры-пыры. А вот в газовой динамике так вообще жесть. И поэтмоу нужно измелчать, а там слишком мелко то нецелесообразно, все дела, и вот начать вводить в адаптивные сетки, как это круто, они могут спасти ситуацию.
}

\texttt{
    Дальше сформулировать цель работы, и кратко рассказать, в каких section'ах что описывается. Это вроде все планы на введение.
}
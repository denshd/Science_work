Большинство моделей классической физики, таких как гидрогазодинамика, описываются начально-краевыми задачами для дифференциальных уравнений в частных производных второго порядка\cite{ТихоновСамарский, ЛандауГидродинамика}.
Нахождение аналитического решения таких задач представляется возможным только в случае простых, канонических областей (таких как круг, шар, прямоугольник), простых начальных и граничных условиях, а также в случае линейных уравнений, описывающих простые физические процессы.

На практике же часто возникают нелинейные задачи, поставленные в областях сложной формы.
Например, задача лазерного термоядерного синтеза (ЛТС), идя которой заключается в быстром нагреве и сжатии термоядерного топлива до температур и плотностей, необходимых для осуществления быстрого и эффективного протекания термоядерных реакций инерциально удерживаемой плазмы.
Процессы распространения тепла в такой системе будут описываться нелинейным уравнением теплопроводности.
Нелинейное уравнение теплопроводности также возникает в, например, задачах о самофокусировки световых пучков в нелинейных средах, эффекте $T$--слоя в низкотемпературной плазме, проблемы безударного сжатия; вообще с необходимостью в любой задаче, в которой присутствует процессы самопроизвольного нарушения симметрии с понижением её степени \cite{ГалактионовКвазилинейное} 

Любой численный метод приближённого решения таких задач использует дискретизацию (то есть переход от бесконечномерного функционального пространства к конечномерному пространству).
Один из основных методов~--- \emph{метод конечных разностей}.
Его основа заключается в том, что исходная непрерывная задача в области $G \subset \mathbb{R}^n$ сводится к \emph{семейству разностных задач}~--- системам конечного числа линейных (в общем случае~--- нелинейных) уравнений на т.н. \emph{разностные функции}~--- функции, заданные на конечном числе точек (именуемых \emph{сетками}), и принимающие значения (приближённые значения решения) на конечном числе точек.
Такие задачи решаются алгоритмически и тем самым могут быть программно реализованы на современных ЭВМ.
Более подробно метод описан в разделе~\ref{sec:MainDiffSchemes}.

Принципиальная возможность применения тех или иных алгоритмов основывается на вопросе об их сходимости, точности и устойчивости.
Так, в работе \cite{СамарскийТеорияРазностныхСхем} даётся обширное описание алгоритмов решения задач на \emph{статических} сетках, исследуются вопросы устойчивости и скорости сходимости.
Более подробное исследование тех же вопросов в случае \emph{неравномерных статических сеток} дан в работе \cite{СамарскийНеравномерныеСетки}.

Как уже отмечалось, в прикладных задачах приходится сталкиваться с квазилинейными уравнениями теплопроводности:
\begin{equation*}
    \frac{\partial u}{\partial t} =
    \sum\limits_{\alpha = 1}^{p} \frac{\partial }{\partial x_{\alpha}} \left[ 
        k_{\alpha}(u) \frac{\partial u}{\partial x_{\alpha}}
     \right]
\end{equation*}
Проблема использования \emph{статических} сеток, то есть сеток, не меняющихся на протяжении всего алгоритмического процесса поиска решения, связана со следующим обстоятельством.
В статьях \cite{ЗельдовичРазрывы, ЕщёРазрывы} показано, что одномерное уравнение теплопроводности в случае зависящего от температуры коэффициента теплопроводности имеет решения, производные которых разрывны в точках обращения в нуль решения $u(x, t)$, при этом поток тепла $k(u) \frac{\partial u}{\partial x}$~--- непрерывен, то есть существует фронт температуры, который, как показано в \cite{ЕщёЕщёРазрывы}, распространяется с конечной скоростью.
Эти \glqq проблемные точки решения\grqq оказываются сильно локализованными:
если для численного решения использовать достаточно грубые сетки, то основные ошибки в приближённом решении будут локализованны именно в окрестностях этих точек.
Конечно, можно использовать более мелкий шаг сетки и улучшить точность решения, ибо, как предсказывает теория \cite{СамарскийТеорияРазностныхСхем}, приближённое решение должно схоидтся к точному при стремлении шага сетки к нулю.

Однако даже на мощных вычислитльных системах расчёт сложных трёхмерных задач со сложной пространственной геометрией требует огромного числа точек сетки, что значительно увеличивает используемую память и расчётное время \cite{АфендиковЛАД}.
Более того, точность решения в области особенностей \emph{существенно} влияет на точность решения во всей остальной области.
Поэтому хотя бы для получения приемлемой кратины решения в целом на всей области без точного учёта особенностей неизбежно приходится сильно измельчать сетку.
Учитывая, что в подобластях гладкого поведения решения просто нет необходимости измельчать сетку настолько сильно, заключаем, что использование классических алгоритмов приводит к тому, что большая часть компьютерных вычислений производится напрасно.

Поэтому для данного класса гидродинамических проблем с локализованными особенностями разрабатывались специальные методы \emph{локально-адаптивных сеток} (\emph{Adaptive mesh refinement}), учитывающие разномасштабное поведение решения.
Например, в работе \cite{СамарскийАдаптивные} предлагалось использовать адаптивную сетку, построение которой производится с помощью соответствующего преобразования координат.
Конкретный вид преобразования задаётся с помощью некоторой функции $Q$, вид которой определяется особенностями решения исследуемой задачи.
Т.к. вид функции $Q$ выбирался вручную в завимимости от конкретной задачи, этот метод не обладал достаточной автономностью.
Многие методы были основаны на геометрической адаптации рассчётных сеток, что, в свою очередь, приводит к трудностям реализации на ЭВМ, поскольку неструктурированные сетки порождают нерегулярный доступ к памяти.
С учётом современного развития массивно-параллельных архитектур процессоров с большим числом ядер, эффективность работы которых зависит в первую очередь от упорядоченности обращений в память, производительность методов с неструктурированными сетками оказывается неудовлетворительной.

\emph{Метод структурированных адаптивных сеток (Block-structured adaptive mesh refinement)} был представлен в работах \cite{berger1982adaptive, berger1989local} применительно к уравнениям гиперболического типа.
Преимущества метода в:
\begin{itemize}
    \item использовании простых прямоугольных областей определённого размещения, удобных для реализации на компьютере
    \item возможности использования архитектуры параллельных вычислений
    \item использовании точно таких же разностных схем, как и для статических декартовых сеток (с некоторыми алгоритмическими модификациями)
\end{itemize}.

\textbf{Целью данной работы} является изучение метода стрктурированных декартовых локально-адаптивных сеток применительно к задачам для уравнения теплопроводности, программная реализация данного метода, сравнение со статическими аналогами.

В разделе \ref{sec:MainDiffSchemes} вводятся основные математические формулировки разностных задач.
В разделе \ref{sec:StaticGrid} описываются алгоритмы решения задач на статических сетках (которые в последствии непосредственно используются при решении методом адаптивных сеток), приводятся примеры решения модельных задач.
В разделе \ref{sec:LAG} приводится описание метода, программной реализации и результатов решения модельных задач.
В работе проведён подробный анализ численного метода разностных схем решения уравнения теплопроводности, исследуются преимущества и недостатки использования равномерных статических сеток в случае квазилинейных уравнений с существенно различающимися масштабами (как пространственными, так и временными) физических процессов.
Разработано программное обеспечение, реализующее рассмотренные схемы численного решения, проверены описанные особенности каждой из схем.
Показана необходимость использования сеток с локальным измельчением, подстраивающихся под особенности решения.
Описаны некоторые теоретические основы данного метода.
Реализован в виде программного кода соответствующий алгоритм, работоспособность которого проверена на нескольких тестовых задачах.
Показаны преимущества использования блочных локально-адаптивных сеток по сравнению с использованием равномерных статических.
\newcommand{\valpha}{v_{(\alpha)}}

Итак, явные схемы обладают быстрой скоростью счёта ($\sim O(N^n)$), в то время как устойчивость таких систем достигается лишь при определённом выборе параметров сетки.
Неявные схемы безусловно устойчивы и имеют больший порядок точности, однако требуют решения системы $N^n$ уравнений, для чего требуется значительно больше вычислительной работы, чем для явной схемы \cite{ТихоновСамарский}.

Для сочетания лучших качеств явных (объём работы $\sim O(N^n)$) и неявных (безусловная устойчивость) схем было предложено несколько \emph{экономичных} схем.
Подбробнее об этом написано в \cite{СамарскийТеорияРазностныхСхем, peaceman1955numerical, douglas1955numerical, яненко1959одном, дьяконов1962разностные, самарский1962одном}.

\emph{Локально-одномерный метод} является универсальным, пригодным для решения квазилинейного уравнения теплопроводности в произвольной области $G$ любого числа измерений.
При использовании в работе блочных локально-адаптивных сеток (раздел \ref{sec:LAG}) используется именно этот метод.
Также будут использоваться прямоугольные области, поэтому формулировка метода будет приведена для таковых.

Итак, рассматриваемую многомерную задачу Дирихле в цилиндре $\bar{Q}_T = \bar{G} \times [0, T]$, $\bar{G} = \prod\limits_{i=1}^{n} [0, L_i]$
\begin{equation*}
    \begin{aligned}
        &\left\{ 
            \begin{aligned}
                & u_t = \sum\limits_{\alpha = 1}^{p} \frac{\partial }{\partial x_{\alpha}} \left[ 
                    k_{\alpha}(u) \frac{\partial u}{\partial x_{\alpha}}
                 \right] + f, && (x, t) \in G \times (0, T]\\
                & \begin{aligned}
                    & u(x, t) = \mu_{-i}(x, t), && x_i = 0\\
                    & u(x, t) = \mu_{i}(x, t), && x_i = L_i
                \end{aligned}, && t \in [0, T)\\
                &u(x, 0) = u_0(x), && x \in \bar{G}
            \end{aligned}
        \right.\\
    \end{aligned}
\end{equation*}
заменяем \emph{цепочкой одномерных} задач \glqq вдоль каждого из напарвлений\grqq:
\begin{equation*}
    \begin{cases}
        \frac{1}{p} \frac{\partial v_{(\alpha)}}{\partial t} = \frac{\partial }{\partial x_{\alpha}} \left[ k_{\alpha}(v_{(\alpha)}) \frac{\partial v_{(\alpha)}}{\partial x_{\alpha}} \right] + f_{\alpha}, &
            x \in G, t \in \Delta_{\alpha} = \left( 
                t_{j + \frac{\alpha - 1}{p}}, 
                t_{j + \frac{\alpha}{p}}
            \right]\\
        \valpha(x, t_{j + \frac{\alpha - 1}{p}}) = v_{(\alpha - 1)}(x, t_{j + \frac{\alpha - 1}{p}}), & x \in G\\
        \valpha(x, t) = \mu_{-\alpha}(x, t), & x_{\alpha} = 0, t \in [t_{j + \frac{\alpha - 1}{p}}, t_{j + \frac{\alpha }{p}}]\\
        \valpha(x, t) = \mu_{\alpha}(x, t), & x_{\alpha} = L_{\alpha}, t \in [t_{j + \frac{\alpha - 1}{p}}, t_{j + \frac{\alpha }{p}}]\\
    \end{cases}
\end{equation*}
\begin{equation*}
    \begin{aligned}
        &\valpha(x, 0) = u_0(x)\\
        & v_{(1)}(x, t_j) = v_{(p)}(x, t_j)\\
        & u(x, t_{j + 1}) = v_{(p)}(x, t_{j + 1}).
    \end{aligned}
\end{equation*}
Каждая из одномерных задач решается неявной двухслойной схемой:
\begin{multline*}
    \frac{v_{(\alpha)}^{j + \frac{\alpha}{p}} - v_{(\alpha)}^{j + \frac{\alpha - 1}{p}}}{\tau} =
    \frac{1}{h_{\alpha}^2} \left[ 
        a_{i_{\alpha} + 1}^{(\alpha)} \left( v_{(\alpha)}^{j + \frac{\alpha - 1}{p}} \right) \cdot \left( 
            v_{(\alpha)i_{\alpha} + 1}^{j + \frac{\alpha}{p}} - v_{(\alpha)i_{\alpha}}^{j + \frac{\alpha}{p}}
         \right)-\right.\\
         \left. - a_{i_{\alpha}}^{(\alpha)} \left(v_{(\alpha)}^{j + \frac{\alpha - 1}{p}}\right) \cdot \left( 
             v_{(\alpha), i_{\alpha}}^{j + \frac{\alpha}{p}} - v_{(\alpha), i_{\alpha} - 1}^{j + \frac{\alpha}{p}}
          \right)
     \right] + f_{\alpha} \left( 
         v_{(\alpha)}^{j + \frac{\alpha - 1}{p}}
      \right), \\
      \text{ где } a_i^{(\alpha)} (y) = k_{\alpha} \left( \frac{y_{i - 1} + y_i}{2} \right), \quad 
      i_{\alpha} \pm 1 = (i_1, \ldots, i_{\alpha - 1}, i_{\alpha} \pm 1, i_{\alpha + 1}, \ldots, i_{p})
\end{multline*}

Напишем подробнее, как выглядит данная схема в случае двумерного квазилинейного уравнения теплопроводности.
Цилиндр $\bar{Q}_T = [0, L_1] \times [0, L_2] \times [0, T]$ дискретизуется равномерной сеткой:
\begin{multline*}
    \omega_{h\tau} = \left\{ 
        (x_i, y_k, t_j) \mid x_i = h_x \cdot (i - 1), y_k = h_y \cdot (k - 1), t_j = \tau \cdot (j - 1),\right.\\
        \left. i = 1,\ldots,N_x,\quad k = 1,\ldots, N_y, \quad j = 1, \ldots, N_t
     \right\}
\end{multline*}
Исходная задача с уравнением
\begin{equation*}
    \frac{\partial u}{\partial t} = \frac{\partial}{\partial x}\left( k_1(u) \frac{\partial u}{\partial x} \right) +
    \frac{\partial}{\partial y} \left( k_2(u) \frac{\partial u}{\partial y} \right) + f
\end{equation*}
заменяется на две одномерные задачи, каждая из которых решается неявной схемой.
Обозначим $w := u^{j + 1/2}$~--- промежуточное значение, получающееся при решении первой из задач на интервале времени $(t_{j}, t_{j + 1/2})$ (вдоль оси $x_1$).
Решение в начальный момент времени $t_1 = 0$ известно из начальных данных:
\begin{equation*}
    u_{i, k}^{1} = u_0(x_i, y_k)\quad \forall i, k
\end{equation*}
Зная решение на $j$--ом временном слое $u_{i, k}^{j}$, решение на $(j + 1)$--ом ищется следующим образом:
сначала решается $N_y$ одномерных задач вдоль оси $x$ ($N_y$~--- для каждого $y_k, k = 1,\ldots,N_y$ независимо):
\begin{equation*}
    \begin{cases}
        \begin{aligned}
            \textstyle\frac{w_{i, k} - u_{i, k}^{j}}{\tau} = &\textstyle \frac{1}{h_x^2} \left[ 
                k_1 \left( 
                    \frac{u_{i, k}^{j} + u_{i + 1, k}^{j}}{2}
                \right) \cdot \left( 
                    w_{i + 1, k} - w_{i, k}
                \right) -
            \right.\\
            &\textstyle \left.
                - k_1 \left( 
                    \frac{u_{i - 1, k}^{j} + u_{i, k}^{j}}{2}
                \right) \cdot \left( 
                    w_{i, k} - w_{i - 1, k}
                \right)
            \right] + \frac{1}{2}\phi_{i, k}^{j + 1/2}
        \end{aligned}, & i = 2, \ldots, N_x - 1\\ 
        w_{1, k} = \mu_{-1}(y_k, t_{j + 1/2})\\
        w_{N_x, k} = \mu_{1}(y_k, t_{j + 1/2})
    \end{cases}
\end{equation*}
Затем решается $N_x$ одномерных задач вдоль оси $y$ ($N_x$~--- для каждого $x_i, i = 1, \ldots, N_x$ независимо):
\begin{equation*}
    \begin{cases}
        \begin{aligned}
            \textstyle\frac{u_{i, k}^{j + 1} - w_{i, k}}{\tau} = &\textstyle \frac{1}{h_y^2} \left[ 
                k_2 \left( 
                    \frac{w_{i, k} + w_{i, k + 1}}{2}
                \right) \cdot \left( 
                    u_{i, k + 1}^{j + 1} - u_{i, k}^{j + 1}
                \right) -
            \right.\\
            &\textstyle \left.
                k_2 \left(
                    \frac{w_{i, k - 1} + w_{i, k}}{2}
                \right) \cdot \left( 
                    u_{i, k}^{j + 1} - u_{i, k - 1}^{j + 1}
                \right)
            \right] + \frac{1}{2}\phi_{i, k}^{j + 1}, 
        \end{aligned} & k = 2, \ldots, N_y - 1\\
        u_{i, 1}^{j + 1} = \mu_{-2}(x_i, t_{j + 1})\\
        u_{i, N_y}^{j + 1} = \mu_2(x_i, t_{j + 1})
    \end{cases}
\end{equation*}
Каждая из одномерных задач решается методом прогонки за $O(N)$, где $N = N_x, N_y$.
И, таким образом, общая сложность такого алгоритма есть $O(N_y \cdot N_x + N_x \cdot N_y) = O(N^2)$, то есть совпадает по скорости решения с явной схемой.

В многомерном случае область $G$ дискретизуется сеткой $\omega_h$, имеющий вдоль каждого направления $N$ точек.
Для каждого значения $\alpha = 1, \ldots, p$ получается $N^{p - 1}$ задач. 
Каждая из них решается методом прогодки за $\sim O(N)$ и, таким образом, число арифметиеских операций, совершаемых в локально-одномерной схеме есть число порядка $\sim O(N^p)$.
Помимо быстрой скорости счёта, ЛОС обладает и существенным достоинством неявных схем.
Как показано в \cite{СамарскийТеорияРазностныхСхем}, схема аппроксимирует исходное уравнение в суммарном смысле:
погрешность аппроксимации $\psi$ для локально-одномерной схемы есть сумма погрешностей аппроксимации $\psi_{\alpha}$ на решении $u = u(x, t)$ для одномерных схем номера $\alpha$:
\begin{equation*}
    \psi = \sum_{\alpha = 1}^{p} \psi_{\alpha} = O(|h|^2 + \tau),
\end{equation*}
ЛОС безусловно устойчива и равномерно сходится.
Это означает, что счёт по данной схеме устойчив даже при очень крупных шагах по времени, что позволяет быстро решать задачи, где не требуется большая точность.

Помимо прочего стоит отметить, что данный метод применим не только к параллелепипедам, но и к произвольным областям $G \subset \mathbb{R}^{p}$, сохраняет порядок точности на неравномерных сетках \cite{СамарскийНеравномерныеСетки}, обладает пониженными (по сравнению с другими схемами) требованиями к объёму перативной памяти, а также допускает распараллеливание: вдоль каждого из напарвлений необходимо решать $N$ систем линейных уравнений, и все они могут решаться независимо друг от друга и, следовательно, параллельно.

Учитывая все преимущества локально-одномерной схемы для уравнения теплопроводности, именно она берётся в основу при рассмотрении локально-адаптивных сеток в разделе \ref{sec:LAG}.
\newcommand{\valpha}{v_{(\alpha)}}

Итак, явные схемы обладают быстрой скоростью счёта ($\sim O(N^n)$), в то время как устойчивость таких систем достигается лишь при определённом выборе параметров сетки.
Неявные схемы безусловно устойчивы и имеют больший порядок точности, однако требуют решения системы $N^n$ уравнений, для чего требуется значительно больше вычислительной работы, чем для явной схемы \cite{ТихоновСамарский}.

Для сочетания лучших качеств явных (объём работы $\sim O(N^n)$) и неявных (безусловная устойчивость) схем было предложено несколько \emph{экономичных} схем.
Подбробнее об этом написано в \cite{СамарскийТеорияРазностныхСхем, peaceman1955numerical, douglas1955numerical, яненко1959одном, дьяконов1962разностные, самарский1962одном}.

\emph{Локально-одномерный метод} является универсальным, пригодным для решения квазилинейного уравнения теплопроводности в произвольной области $G$ любого числа измерений.
При использовании в работе блочных локально-адаптивных сеток \ref{sec:LAG} используется именно этот метод.
Также будут использоваться прямоугольные области, поэтому формулировка метода будет приведена для таковых.

Итак, рассматриваемую многомерную задачу Дирихле в цилиндре $\bar{Q}_T = \bar{G} \times [0, T]$, $\bar{G} = \prod\limits_{i=1}^{n} [0, L_i]$
\begin{equation*}
    \begin{aligned}
        &\left\{ 
            \begin{aligned}
                & u_t = \sum\limits_{\alpha = 1}^{p} \frac{\partial }{\partial x_{\alpha}} \left[ 
                    k_{\alpha}(u) \frac{\partial u}{\partial x_{\alpha}}
                 \right] + f, && (x, t) \in G \times (0, T]\\
                & \begin{aligned}
                    & u(x, t) = \mu_{-i}(x, t), && x_i = 0\\
                    & u(x, t) = \mu_{i}(x, t), && x_i = L_i
                \end{aligned}, && t \in [0, T)\\
                &u(x, 0) = u_0(x), && x \in \bar{G}
            \end{aligned}
        \right.\\
    \end{aligned}
\end{equation*}
заменяем \emph{цепочкой одномерных} задач \glqq вдоль каждого из напарвлений\grqq:
\begin{equation*}
    \begin{cases}
        \frac{1}{p} \frac{\partial v_{(\alpha)}}{\partial t} = \frac{\partial }{\partial x_{\alpha}} \left[ k_{\alpha}(v_{(\alpha)}) \frac{\partial v_{(\alpha)}}{\partial x_{\alpha}} \right] + f_{\alpha}, &
            x \in G, t \in \Delta_{\alpha} = \left( 
                t_{j + \frac{\alpha - 1}{p}}, 
                t_{j + \frac{\alpha}{p}}
            \right]\\
        \valpha(x, t_{j + \frac{\alpha - 1}{p}}) = v_{(\alpha - 1)}(x, t_{j + \frac{\alpha - 1}{p}}), & x \in G\\
        \valpha(x, t) = \mu_{-\alpha}(x, t), & x_{\alpha} = 0, t \in [t_{j + \frac{\alpha - 1}{p}}, t_{j + \frac{\alpha }{p}}]\\
        \valpha(x, t) = \mu_{\alpha}(x, t), & x_{\alpha} = L_{\alpha}, t \in [t_{j + \frac{\alpha - 1}{p}}, t_{j + \frac{\alpha }{p}}]\\
    \end{cases}
\end{equation*}
\begin{equation*}
    \begin{aligned}
        &\valpha(x, 0) = u_0(x)\\
        & v_{(1)}(x, t_j) = v_{(p)}(x, t_j)\\
        & u(x, t_{j + 1}) = v_{(p)}(x, t_{j + 1}).
    \end{aligned}
\end{equation*}
Каждая из одномерных задач решается неявной двухслойной шеститочечной схемой с весом $\sigma_{\alpha}$.
Пускай область $G$ дискретизуется сеткой $\omega_h$, имеющий вдоль каждого направления $N$ точек.
Для каждого значения $\alpha = 1, \ldots, p$ получается $N^{p - 1}$ задач. 
Каждая из них решается (методом прогодки как неявная одномерная схема) за $\sim O(N)$.
Таким образом, 



% \begin{equation*}
%     \frac{1}{p} \frac{\partial v_{(\alpha)}}{\partial t} =
%     \frac{\partial }{\partial x_{\alpha}} \left[ 
%                     k_{\alpha}(v_{(\alpha)}) \frac{\partial v_{(\alpha)}}{\partial x_{\alpha}} \right] + f_{\alpha},
% \end{equation*}
% \begin{equation*}
%     \text{где } t \in \Delta_{\alpha} = \left( 
%         t_{j + \frac{\alpha - 1}{p}}, 
%         t_{j + \frac{\alpha}{p}}
%     \right], \quad \alpha = 1, \ldots, p; \quad \sum_{\alpha = 1}^{p} f_{\alpha} = f
% \end{equation*}
% с граничными условиями:
% \begin{equation*}
%     \begin{cases}
%         \valpha(x, t) = \mu_{-\alpha}(x, t), & x_{\alpha} = 0, t \in \Delta_{\alpha}
%         \valpha(x, t) = \mu_{\alpha}(x, t), & x_{\alpha} = L_{\alpha}, t \in \Delta_{\alpha}
%     \end{cases}, \quad \alpha = 1, \ldots, p
% \end{equation*}
% и начальными условиями:
% \begin{equation*}
%     \begin{cases}
        
%     \end{cases}
% \end{equation*}

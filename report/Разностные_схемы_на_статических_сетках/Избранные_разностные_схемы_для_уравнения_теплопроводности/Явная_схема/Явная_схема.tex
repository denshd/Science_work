\begin{figure}[h]
    \centering
        \begin{tikzpicture}
            \filldraw[black] (0, 0) circle(2pt) node[anchor=south west] {$(x_i, t_j)$};
            \draw (-2.5, 0) node[anchor=north] {$(x_{i - 1}, t_j)$} -- (2.5, 0) node[anchor=north] {$(x_{i + 1}, t_j)$};
            \filldraw[black] (-2.5, 0) circle(2pt);
            \filldraw[black] (2.5, 0) circle(2pt);
            \filldraw[black] (0, 2.5) circle(2pt);
            \draw (0, 0) -- (0, 2.5) node[anchor=south]{$(x_i, t_{j + 1})$};
        \end{tikzpicture}
    \caption{Пятиточечный шаблон явной схемы}
    \label{fig:ExplicitTemplate}
\end{figure}
Шаблон явной схемы нагляден в одномерном случае (\seefigref{fig:ExplicitTemplate}).
Множество $\{(x, t) \in \omega_h \mid t = t_j\}$ будем называть $j$--ым временным слоем.
Оператор $\frac{\partial u}{\partial t}$ аппроксимируется, используя значения функции на $(j + 1)$--ом временном слое и $j$--ом временном слое, а оператор Лапласа аппроксимируется, используя значения только на $j$--ом временном слое:
\begin{equation*}
    L = \frac{\partial u}{\partial t} - \Delta u \mapsto 
    L_{h\tau} = \frac{u_i^{j + 1} - u_i^j}{\tau} - 
    \frac{u_{i + 1}^j - 2u_i^j + u_i^j}{h^2} + \phi_i^j,\quad \phi_i^j = f(x_i, t_j)
\end{equation*}
В многомерном случае:
\begin{multline*}
    L \mapsto 
    \frac{u_i^{j + 1} - u_i^j}{\tau} + \sum\limits_{k = 1}^{n}
    \frac{u_{i+}^j - 2u_i^j + u_{i-}^j}{h_k^2} + \phi_i^j,\text{ где }\\
    i_{\pm} = (i_1, \ldots, i_{k - 1}, i_k \pm 1, i_{k + 1}, \ldots, i_n)
\end{multline*}
Такой \glqq запаздывающий\grqq выбор шаблона позволяет явно выразить значения функции на $(j + 1)$--ом временном слое через значения на $j$--ом временном слое:
\begin{equation}\label{eq:explicit_scheme}
    u_i^{j + 1} = u_i^j + \tau \sum\limits_{k = 1}^{n}
    \frac{u_{i+}^j - 2u_i^j + u_{i-}^j}{h_k^2} + \phi_i^j
\end{equation}
(отсюда и название схемы). Значения функции на 1--ом временном слое следуют из начальных условий: $u_i^1 = u_0(x_i)$, а граничные условия дают замкнутую систему уравнений:
\begin{equation*}
    \begin{cases}
        u_i^{j} = u_i^j + \tau \sum\limits_{k = 1}^{n} \frac{u_{i+}^j - 2u_i^j + u_{i-}^j}{h_k^2} + \phi_i^j, & i_k = 2, \ldots, N_k - 1\\
        u_i^1 = u_0(x_i), & i_k = 1, \ldots, N_k\\
        u_{i_k}^{j} = \mu_{-\alpha}(x_i, t_j), & i_{k \ne \alpha} = 1, \ldots, N_k;\,\, i_{\alpha} = 0\\
        u_{i_k}^{j} = \mu_{+\alpha}(x_i, t_j), & i_{k \ne \alpha} = 1, \ldots, N_k;\,\, i_{\alpha} = N_{\alpha}
    \end{cases}
\end{equation*}
Одно из преимуществ такой схемы~--- простота программной реализации.
Так, задав начальные условия на 1--ом временном слое и граничные значения на всех последующих $j > 1$, явно считаются значения на всех последующих слоях.
В разделе \ref{sec:StaticCode} представлен код и описание программы, реализующей явную разностную схему для одномерного уравнения теплопроводности.
Другое преимущество схемы~--- скорость счёта.
Число арифметических операций, необходимых для расчёта значений функции на одном временном слое порядка $O(N_x)$, где $N_x$~--- число точек по оси $x$ (в многомерном случае~--- $O(\prod N_{\alpha})$).
Ещё одно преимущество схемы~--- возможность параллельного счёта.
Из формулы \eqref{eq:explicit_scheme} видно, что расчёт значений $u_i^{j + 1}$ не зависит от $i$: зависимость от $i$ присутствует в правой части, но эти значения в $j$--ый момент времени уже предполагаются известными, поэтому значения на новом временном слое $u_i^{j + 1}$ и $u_k^{j + 1}$ для разных $i$ и $k$ могут считаться независимо друг от друга.

Схема обладает лишь первым порядком сходимости: $O(h + \tau)$, что заставляет использовать достаточно мелкую сетку.
Более того, схема имеет существенный недостаток~--- она устойчива и сходится к решению лишь при условии
$
\tau \le \frac{h^2}{2n}
$.

Как показано в \cite{СамарскийТеорияРазностныхСхем}, в случае квазилинейных уравнений условие устойчивости приобретает вид:
\begin{equation*}
    \tau \le \frac{h^2 }{2 \cdot \max{k(u)}}
\end{equation*}
Откуда видно, что если $k(u)$ является быстроменяющейся функцией (например $k \sim u^\sigma$), то \emph{использование явных схем нецелесообразно}, поскольку требуется очень мелкий шаг $\tau$ по времени.
Поэтому для квазилинейных уравнений теплопроводности применяются преимущественно \emph{неявные} схемы.
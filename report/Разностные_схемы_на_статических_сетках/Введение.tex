Основное преимущество статических равномерных сеток~--- 
\begin{itemize}
    \item относительно простая реализация в виде программного кода\footnote{Так, например, реализация явной схемы может занимать не более 10 строк кода (подробнее см.~раздел~\ref{sec:StaticCode})};
    \item удобное и простое представление данных в программе. Так, например, рассматривая задачу для двумерного нестационарного уравнения теплопроводности, результаты вычислений программы могут хранится в трёхмерном массиве (в отличие от алгоритмов на нестатических и неравномерных сетках, где используются более сложные структуры, см. подробнее в~разд.~\ref{sec:LAG});
    \item существенное упрощение формул и доказательств сходимости, устойчивости получающихся разностных схем.
\end{itemize}
Всюду далее под равномерной статической сеткой будем понимать:
\begin{equation*}
    \omega_{h\tau} := \left\{ 
        ({x_i}_k = k \cdot h_i, t_j = j \cdot \tau) \mid 
        i = 1, \ldots, n; k = 1, \ldots, N_i
     \right\}
\end{equation*}
Как указывалось выше, свойства разностной схемы зависят и от выбора аппроксимирующего оператора $L_h$.
В \cite{СамарскийТеорияРазностныхСхем} исследуется множество различных схем.
Далее приводятся избранные схемы, каждая из которых обладает отличительной особенностью, для задачи Дирихле для линейного и квазилинейного уравнений теплопроводности.
\begin{equation*}
    \begin{aligned}
        &\left\{ 
            \begin{aligned}
                & u_t = \Delta u + f, && (x, t) \in G \times (0, T)\\
                & \begin{aligned}
                    & u(x, t) = \mu_{-i}(x, t), && x_i = 0\\
                    & u(x, t) = \mu_{i}(x, t), && x_i = L_i
                \end{aligned}, && t \in [0, T)\\
                &u(x, 0) = u_0(x), && x \in \bar{G}
            \end{aligned}
        \right.\\
        &\text{где } \bar{G} = \prod\limits_{i=1}^{n} [0, L_i].
    \end{aligned}
\end{equation*}
% Одномерная постановка задачи:
% \begin{equation*}
%     \left\{
%     \begin{aligned}
%         &u_t = u_{xx} + f(x, t), && (x, t) \in (0, 1)\times(0, T)\\
%         &\begin{aligned}
%         &u(0, t) = \mu_1(t)\\
%         &u(1, t) = \mu_2(t)
%         \end{aligned}, && t\in [0, 1)\\
%         &u(x, 0) = u_0(x), && x \in [0, T]
%     \end{aligned}\right.
% \end{equation*}
% \usemintedstyle[julia]{native}

Реализация вышеобозначенных разностных схем производилась с помощью языка программирования \texttt{Julia}\cite{bezanson2017julia}. 

В качестве иллюстрации простоты явной разностной схемы, в листинге \ref{listing:explicit_scheme} приведена её реализация для одномерного линейного уравнения теплопроводности.
\begin{listing}
    \begin{minted}[
        frame=lines,
        framesep=2mm,
        fontsize=\footnotesize,
        linenos=true,
        autogobble
        ]{julia}
    function explicit_scheme(f, u0, mu1, mu2, T, Nx, Nt)
        x = range(0, 1, length = Nx) # Точки секти по оси `x`
        t = range(0, T, length = Nt) # Точки сетки по оси `t`
        c = step(t) / step(x)^2 # Коэффициент Куранта
    
        u = zeros(Nx, Nt) # Матрица, хранящая решение разностной задачи
    
        u[:, 1] .= u0.(x) # Учёт начальных условий
        u[1, :] .= mu1.(t) # Учёт граничных условий на левом конце
        u[end, :] .= mu2.(t) # Учёт граничных условий на правом конце
    
        for j in 1:(Nt - 1) # Цикл по всем слоям
            for i in 2:(Nx - 1)
                u[i, j + 1] = u[i, j] + c * (u[i + 1, j] - 2 * u[i, j] + u[i - 1, j]) + 
                f(x[i], t[j])
            end
        end
    
        return u
    end
    \end{minted}
    \caption{Реализация явной схемы для уравнения $u_t = u_{xx} + f$}
    \label{listing:explicit_scheme}
\end{listing}
Заметим, что схема реализована для уравнения $u_t = u_{xx} + f$ на отрезке $x \in [0, 1]$, поскольку более общий вид уравнения $u_t = a^2 u_{xx} + \tilde{f}$ на отрезке $[a, b]$ сводится к первому линейной заменой $ t \mapsto \tilde{t} = \frac{(b - a)^2}{a^2} t, \quad x \mapsto \tilde{x} = \frac{x - a}{b - a}$.
Работоспособность программы была проверена на некоторых тестовых задачах \cite{горюнов2015методы}.
Так, для задачи
\begin{equation*}
    \begin{cases}
        u_t = u_{xx} + f, & (x, t) \in (0, 1) \times (0, T]\\
        u(x, 0) = u_0(x), & x \in [0, 1]\\
        \begin{aligned}
            & \textstyle u(0, t) = \mu_1(t)\\
            & \textstyle u(1, t) = \mu_2(t)
        \end{aligned}, & t \in [0, T]
    \end{cases}
\end{equation*}
расчитано решение для $T = 0.05$ для различного набора сеток с числом точек по оси времени $N_t = 25001$ и числом точек по оси $x$ в пределах от $N_x = 10$ до $N_x = 500$.
Заметим, что выбор такого, казалось, необоснованно большого числа точек по оси времени диктуется неустойчивостью явной разностной схемы: из условия устойчивости $\frac{\Delta t}{(\Delta x)^2} < \frac{1}{2}$ следует, что для $N_x = 500$ требуется $N_t > 25000$.
В таких простых, тестовых задачах такое соотношение ещё приемлемо, и современные компьютеры позволяют получить ршение достаточно быстро.
Однако в более сложных, многомерных задачах с резкими неоднородностями, требующими достаточно мелкого пространственного шага, это условие становится трудно выплонимым (такая точность по оси времени просто не требуется).
Явная схема была реализована только для линейного уравнения.
Как уже отмечалось выше, применения явной схемы нецелесообразно для квазилинейных уравнений.

На (бла-бла) представлено численное решение задачи как поверхность-график зависимости $u_h(x, t)$, полученное явной схемой для $N_x = 501$ и $N_t = 25001$.
На (бла-бла) представлен график зависимости аналитического решения задачи, полученного методом разделения переменных Фурье:
\begin{equation*}
    u(x, t) = \frac{1}{2}e^{-(3\pi)^2 t}\sin (3\pi x) - \frac{48}{\pi^2}\sum\limits_{m=1}^{\infty} \frac{m}{(9 - 4m^2)^2}e^{-(2\pi m)^2 t}\sin (2\pi m x)
\end{equation*}
На (бла-бла) представлен график зависимости локальной ошибки $|u(x, t) - u_h(x, t)|$.
Для проверки утверждения о точности аппроксимации разностной схемы также построен график зависимости нормы разности аналитического и численного решения как функции от шага $h$.
Бла-бла-бла после построения.

Реализация неявной схемы требует написания алгоритма прогонки для решения систем линейных уравнений с трёхдиагональной матрицей.
В листинге \ref{listing:tridiagonal_algorithm} представлен код алгоритма прогонки.
\begin{listing}
    \begin{minted}[
        frame=lines,
        framesep=2mm,
        fontsize=\footnotesize,
        linenos,
        autogobble
        ]{julia}
    # Оптимальный по памяти алгоритм прогонки в случае, если его нужно запускать много раз
    function tridiagonal_algorithm!(
            a::Vector{<:T}, b::Vector{<:T}, x::Vector{<:T},
            A::Vector{<:V}, B::Vector{<:V}, C::Vector{<:V}, D::Vector{<:V}
        ) where {T <: Number, V <: Number}
    
        a[1] = - C[1] / B[1]
        b[1] = D[1] / B[1]
        @inbounds for i=2:(length(B)-1)
            a[i] = - C[i] / (B[i] + A[i-1] * a[i-1])
            b[i] = (D[i] - A[i-1] * b[i-1]) / (B[i] + A[i-1] * a[i-1])
        end
        b[end] = (D[end] - A[end] * b[end-1]) / (B[end] + A[end] * a[end-1])
    
        # обратный ход
        x[end] = b[end]
        for i = length(B)-1:-1:1
            x[i] = a[i]*x[i+1] + b[i]
        end
    end
    \end{minted}
    \caption{Реализация алгоритма прогонки}
    \label{listing:tridiagonal_algorithm}
\end{listing}
Дополнительные переменные $\alpha, \beta, x$ передаются в функцию \texttt{tridiagonal\_algorithm} с целью экономии памяти.
Решение уравнения теплопроводности требует многократного выполнения алгоритма прогонки, поэтому общие временные переменные выделяются отдельно и сокращают количество ненужных аллокаций.
В листинге \ref{listing:implicit_scheme} приведена реализация неявной схемы для одномерного квазилинейного уравнения теплопроводности.
\begin{listing}
    \begin{minted}[
        frame=lines,
        framesep=2mm,
        fontsize=\footnotesize,
        linenos,
        autogobble
        ]{julia}
    function implicit_scheme(k, f, u0, mu1, mu2, T, Nx, Nt)
        x = range(0, 1, length = Nx) # Точки секти по оси `x`
        dx = step(x)
        t = range(0, T, length = Nt) # Точки сетки по оси `t`
        dt = step(t)
    
        u = zeros(Nx, Nt) # Матрица, хранящая решение разностной задачи
        u[:, 1] .= u0.(x) # Учёт начальных условий
        u[1, :] .= mu1.(t) # Учёт граничных условий на левом конце
        u[end, :] .= mu2.(t) # Учёт граничных условий на правом конце
    
        # Временные переменные для алгоритма прогонки
        A, C = (zeros(Nx) for i in 1:2)
        B, D, a, b, x = (zeros(Nx) for i in 1:5)
        B[1] = B[end] = 1
        C[1] = A[end] = 0
    
        for j in 1:(Nt - 1) # Цикл по временным слоям
            # Обновление коэффициентов тридиагональной матрицы
            D[1] = u[1, j + 1]
            D[end] = u[end, j + 1]
            for i = 2:(Nx - 1)
                A[i - 1] = -dt * k(0.5 * (u[i - 1, j] + u[i, j]))
                B[i] = dx^2 + dt * (
                    k(0.5 * (u[i, j] + u[i + 1, j])) +
                    k(0.5 * (u[i - 1, j] + u[i, j]))
                )
                C[i] = -dt * k(0.5 * (u[i, j] + u[i + 1, j]))
                D[i] = dx^2 * u[i, j] + dt * dx^2 * f(x[i], t[j])
            end
            tridiagonal_algorithm!(a, b, x, A, B, C, D) # Метод прогонки
            u[:, j + 1] .= x
        end
    
        return u
    end
    \end{minted}
    \caption{Реализация неявной схемы для уравнения $u_t = (ku_x)_x + f$}
    \label{listing:implicit_scheme}
\end{listing}
Работоспособность программы проверена на тех же тестовых задачах, что и для явной схемы.
Для проверки правильности учёта квазилинейности уравнения программа проверена на задаче, рассмотренной в статье~\cite{самарский1963примеры}:
\begin{equation*}
    \left\{
        \begin{aligned}
            &\frac{\partial u}{\partial t} = \frac{\partial }{\partial x}\left( \frac{1}{2}u^2 \frac{\partial u}{\partial x} \right), && (x, t) \in (0, 1)\times(0.1, 0.4)\\
            &\begin{aligned}
            &u(0, t) = 10\sqrt{t}\\
            &u(1, t) = 0
            \end{aligned}, && t\in [0.1, 0.4]\\
            &u(x, 0.1) = \left\{ \begin{aligned}
                &2\sqrt{5}\sqrt{0.5 - x}, && x \le 0.5\\
                & 0, && x\ge 0.5
            \end{aligned}\right. , && x \in [0, 1]
        \end{aligned}
    \right.
\end{equation*}
На графике (бла-бла) приведено численное решение, полученное неявной схемой на сетке с $N_x = 31$, $N_t = 50$ и аналитическое решение задачи:
\begin{equation*}
    u(x, t) = \left\{
        \begin{aligned}
            &\left[ \sigma c\varkappa_0^{-1} (ct + x_1 - x) \right]^{1/\sigma}, && x \le x_1 + ct\\
            &0, && x \ge x_1 + ct
        \end{aligned}
    \right.
\end{equation*}
где $\sigma = 2$, $\varkappa_0 = 0.5$, $x_1 = 0$, $c = 5$. 
Также на (бла-бла) представлен график зависимости локальной ошибки.
Видно, что основная погрешность сосредоточена в точках \emph{температурного фронта волны}: всюду, кроме некоторых точек около фронта, отклонение приближённого решения от аналитического не превосходит $2\cdot 10^{-3}$.

Однако для достижения такой же точности и \emph{вблизи особых точек} решения необходима достаточно мелкая сетка.
На (бла-бла) график решения и ошибки для $N_x = (\text{посчитать    })$.

Данный пример хорошо демонстрирует причину возникновения метода локально-адаптивных сеток.
Для получения точного решения вблизи особенности (фронта волны), необходима достаточно мелкая сетка $N \sim \text{число}$.
Напротив же, в остальных точках решение ведёт себя достаточно гладким и полчается точным уже при мелкой сетки $N \sim \text{число}$.
В одномерном случае с одной особой точкой эта ситуация может показаться не столь плохой.
Для многомерных же задач с большим числом особенностей и сильно меняющимися масштабами решения, где для получения заданной точности приходится использовать минимально необходимый шаг сетки во всей области сразу, число ненужных операций алгоритма значительно возрастает до недопустимых значений. 

В многомерных задачах, как уже отмечалось, получающиеся системы будут не трёхдиагональными, однако по-прежнему останутся достаточно разреженными (см. картинку и тут ссылка на картинку с spy matrix для 2d).
Реализация такой схемы заключается в прямом создании матрицы системы на каждом временном слое и решении с помощью общего алгоритма исключения переменных Гаусса.
\textbf{Вот тут вопрос: стоит ли в Приложении прикладывать листинг реализации этой схемы, и графики, демонстрирующие её работу и медлительность?}.

Локально-одномерная схема в сущности решает $N^{p - 1}$ одномерных задач, поэтому её реализация похожа на неявную схему, но с некоторыми отличиями.
В листинге (бла-бла) приведёт код этой схемы для двумерного квазилинейного уравнения теплопроводности (на случай больших размерностей код модифицируется очевидным образом (\textbf{стоит ли приводить для трёхмерного случая код в приложении?})).
Работоспособность проверена на задаче, рассмотренной в статье \cite{самарский1963примеры}:
\begin{equation*}
    \textstyle
    \begin{cases}
        u_t = (4u^4 u_x)_x + \left( \frac{1}{4} u^2 u_y \right)_y, \quad (x, y, t) \in (0, 30) \times (0, 20) \times (0, 50)\\
        u(x, y, 0) = 0,\\
        u(0, y, t) = \mu_{-1}(y, t) = \begin{cases}
            \scriptstyle\frac{1}{2}\sqrt{-1 + \sqrt{1 + 16(t - 2y)}}, & t > 2y\\
            \hfill 0, & t < 2y
        \end{cases}\\
        u(30, y, t) = \mu_1(y, t) = \begin{cases}
            \scriptstyle\frac{1}{2}\sqrt{-1 + \sqrt{1 + 16(t - 30 - 2y)}}, & t > 30 + 2y\\
            \hfill 0, & t < 30 + 2y
        \end{cases}\\
        u(x, 0, t) = \mu_{-2}(x, t) = \begin{cases}
            \scriptstyle\frac{1}{2}\sqrt{-1 + \sqrt{1 + 16(t - x)}}, & t > x\\
            \hfill 0, & t < x
        \end{cases}\\
        u(x, 20, t) = \mu_2(x, t) = \begin{cases}
            \scriptstyle\frac{1}{2}\sqrt{-1 + \sqrt{1 + 16(t - x - 40)}}, & t > x + 40\\
            \hfill 0, & t < x + 40
        \end{cases}
    \end{cases}
\end{equation*}
\textbf{Дальше только построение графиков и обсуждение результатов этих графиков, плавный переход к адаптивным сеткам}